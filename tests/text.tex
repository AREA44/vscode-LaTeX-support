\documentclass[a4paper,12pt]{article}
\begin{document}

%Title
\title{My First Document}
\author{My Name}
\date{\today}
\maketitle

%Sections
\section{Introduction}
This is the introduction.
\section{Methods}
\subsection{Stage 1}
The first part of the methods.
\subsection{Stage 2}
The second part of the methods.
\section{Results}
Here are my results.

%Labelling
Referring to section \ref{sec1} on page \pageref{sec1}

%Table of Contents
\pagenumbering{roman}
\tableofcontents
\newpage
\pagenumbering{arabic}

%Font Effects
\textit{words in italics}
\textsl{words slanted}
\textsc{words in smallcaps}
\textbf{words in bold}
\texttt{words in teletype}
\textsf{sans serif words}
\textrm{roman words}
\underline{underlined words}

%Coloured Text
{\color{colour_name}text}

%Font Sizes
{\tiny tiny words}
{\scriptsize scriptsize words}
{\footnotesize footnotesize words}
{\small small words}
{\normalsize normalsize words}
{\large large words}
{\Large Large words}
{\LARGE LARGE words}
{\huge huge words}

%Lists
\begin{itemize}
\item[-] First thing
\item[+] Second thing
\begin{itemize}
\item[Fish] A sub-thing
\item[Plants] Another sub-thing
\end{itemize}
\item[Q] Third thing
\end{itemize}

%Tables
\begin{tabular}{|l|l|}
Apples & Green \\
Strawberries & Red \\
Oranges & Orange \\
\end{tabular}
\begin{tabular}{rc}
Apples & Green \\
\hline
Strawberries & Red \\
\cline{1-1}
Oranges & Orange \\
\end{tabular}
\begin{tabular}{|r|l|}
\hline
8 & here’s \\
\cline{2-2}
86 & stuff \\
\hline \hline
2008 & now \\
\hline
\end{tabular}

%Figures
\begin{figure}[h]
\centering
\includegraphics[width=1\textwidth]{myimage}
\caption{Here is my image}
\label{image-myimage}
\end{figure}

%Practical
\begin{figure}[h!]
\centering
\includegraphics[width=1\textwidth]{ImageFilename}
\caption{My test image}
\end{figure}

%Inserting Equations
$1+2=3$
$$1+2=3$$
\begin{equation}1+2=3\end{equation}
\begin{eqnarray}
  a & = & b + c \\
    & = & y - z
\end{eqnarray}

%Mathematical Symbols
%Powers & Indices
$n^2$
$2_a$ 
%Fractions
$$\frac{a}{3}$$
$$\frac{y}{\frac{3}{x}+b}$$
%Roots
$$\sqrt{y^2}$$
$$\sqrt[x]{y^2}$$
%Sums & Integrals
$$\sum_{x=1}^5 y^z$$
$$\int_a^b f(x)$$
%Greek letters
$\alpha$ = α
$\beta$ = β
$\delta, \Delta$ = δ, ∆
$\theta, \Theta$ = θ, Θ
$\mu$ = µ
$\pi, \Pi$ = π, Π
$\sigma, \Sigma$ = σ, Σ
$\phi, \Phi$ = φ, Φ
$\psi, \Psi$ = ψ, Ψ
$\omega, \Omega$ = ω, Ω

%Comments
\begin{comment}
This is a comments.
\end{comment}

%Minted
\begin{minted}{python}
import numpy as np
 
def incmatrix(genl1,genl2):
    m = len(genl1)
    n = len(genl2)
    M = None #to become the incidence matrix
    VT = np.zeros((n*m,1), int)  #dummy variable
 
    #compute the bitwise xor matrix
    M1 = bitxormatrix(genl1)
    M2 = np.triu(bitxormatrix(genl2),1) 
 
    for i in range(m-1):
        for j in range(i+1, m):
            [r,c] = np.where(M2 == M1[i,j])
            for k in range(len(r)):
                VT[(i)*n + r[k]] = 1;
                VT[(i)*n + c[k]] = 1;
                VT[(j)*n + r[k]] = 1;
                VT[(j)*n + c[k]] = 1;
 
                if M is None:
                    M = np.copy(VT)
                else:
                    M = np.concatenate((M, VT), 1)
 
                VT = np.zeros((n*m,1), int)
 
    return M
\end{minted}

%Listing
\begin{lstlisting}{python}
import numpy as np
 
def incmatrix(genl1,genl2):
    m = len(genl1)
    n = len(genl2)
    M = None #to become the incidence matrix
    VT = np.zeros((n*m,1), int)  #dummy variable
 
    #compute the bitwise xor matrix
    M1 = bitxormatrix(genl1)
    M2 = np.triu(bitxormatrix(genl2),1) 
 
    for i in range(m-1):
        for j in range(i+1, m):
            [r,c] = np.where(M2 == M1[i,j])
            for k in range(len(r)):
                VT[(i)*n + r[k]] = 1;
                VT[(i)*n + c[k]] = 1;
                VT[(j)*n + r[k]] = 1;
                VT[(j)*n + c[k]] = 1;
 
                if M is None:
                    M = np.copy(VT)
                else:
                    M = np.concatenate((M, VT), 1)
 
                VT = np.zeros((n*m,1), int)
 
    return M
\end{lstlisting}

A sentence of text.
\end{document}